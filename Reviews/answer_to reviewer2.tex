\documentclass{article}
\usepackage{authblk}
\usepackage[utf8]{inputenc}
\usepackage[T1]{fontenc}
\usepackage[ngerman]{babel}
\usepackage[normalem]{ulem}

%============================================ Define Titlepage & packages =============================================%

\title{Review - revision}

\author{A double-sampling extension of the German National Forest Inventory for design-based small area estimation on forest district levels\\
Andreas Hill, Daniel Mandallaz, Joachim Langshausen}

\usepackage{fancyhdr}     
\usepackage{amsmath} %Paket für erweiterte math. Formeln
\usepackage[labelfont=bf]{caption}
\usepackage[font=footnotesize]{caption}
\usepackage[font=footnotesize]{subcaption}
\usepackage{graphicx}
\usepackage{caption}
\usepackage{subcaption}
\usepackage[final]{pdfpages}
\usepackage{color}

\usepackage{geometry}
\geometry{
	a4paper,
	left=25mm,
	right=25mm,
	top=30mm,
	bottom=30mm
}

\setlength{\parindent}{0em} % Einzug bei neuen Absätzen

%------------------------------------------------------------------------------------------------%
% -------------------------------------- Main Document------------------------------------------ %

\begin{document}

%------------------------------------------------------------------------------------------------%
% -------------------------------------- Tex Settings ------------------------------------------ %

\maketitle
\thispagestyle{empty}
\newpage

\pagenumbering{arabic}
\setcounter{page}{1}

\pagestyle{fancy} %Kopfzeile und Fusszeile
\fancyfoot[C]{\thepage}
\setlength{\headsep}{15mm}

\definecolor{mybrown}{rgb}{0.6, 0.15, 0.1}
\definecolor{amaranth}{rgb}{0.9, 0.17, 0.31}
\definecolor{mygreen}{rgb}{0.1, 0.4, 0.4}
\newcommand{\answer}[1]{\small \color{mybrown}{#1} \color{black}}
\newcommand{\note}[1]{\textit{\small \color{amaranth} \textbf{Note:} #1} \color{black}}
\newcommand{\todo}[1]{\color{red}{#1} \color{black}}
\newcommand{\answerfin}[1]{\small \color{mygreen}{#1} \color{black}}


%------------------------------------------------------------------------------------------------%
% ---------------------------------- Reviewer 2 ------------------------------------------------ %

\section*{Reviewer 2:}

General comment on publishing the article in Remote Sensing:

We decided to hand in our article to be published in the \textit{Remote Sensing} special issue \textit{Remote sensing based Forest Inventories from Landscape to Global Scale}, since multiple of the mentioned topics for articles of the special issue are covered by our article (such as 'Application of remote-sensing techniques for monitoring forest attributes at the forest enterprise level', 'Application of remote-sensing techniques for monitoring forest attributes for national forest inventories', 'Large scale in-situ data requirement and sampling design for monitoring forest ecosystems with remote-sensing techniques').


\begin{enumerate}

  % 01) ++++++++++++++++++++++++++ %
  \item \textit{L28: “entire forest state area of Germany” I understand what you mean, but it may be misleading as readers might think about the area owned by the German state forest enterprise (Bundesforst). I would suggest to rephrase “forested area of Germany”.}
  
  \answerfin{Has been changed accordingly.}
  % --------------------------- %
  
  
  % 02) ++++++++++++++++++++++++++ %
  \item \textit{L29: 114'191 ha – A wrong unit was used here, its 114 191 km$^2$ (or 11 419 124 ha)..}
  
  \answerfin{Has been corrected accordingly.}  
  % --------------------------- %
  
  
   % 03) ++++++++++++++++++++++++++ %
  \item \textit{L36: “standwise forest inventories (SFI)”?}
  
  \answerfin{Has been changed accordingly.}  
  % --------------------------- %
  
  
  % 04) ++++++++++++++++++++++++++ %
  \item \textit{L38-41: Please add the information that the FDI of Lower-Saxony (Betriebsinventur, BI) is only done for forests owned by the state forest enterprise (Lower Saxony State Forests = Niedersächsische Landesforsten, NLF)}
  
  \answerfin{The information has been added.}  
  % --------------------------- %
  
  
  % 05) ++++++++++++++++++++++++++ %
  \item \textit{L41 As far as I remember, [4] is an article written in German. If so, please indicate in the reference list – it is annoying for readers to search for papers that turn out to be written in a language they possibly do not understand.}
  
  \answerfin{We now give the original title in German.}  
  % --------------------------- %
  
  % 06) ++++++++++++++++++++++++++ %
  \item \textit{L42: As far as I remember, v.Lüpke’s dissertation [5] is a cumulative work. If so, please cite the original articles and not the compilation.}
  
  \answerfin{Has been changed accordingly.}  
  % --------------------------- %
  
  
  % 07) ++++++++++++++++++++++++++ %
  \item \textit{L43: I would suggest to cite the classic textbooks of Cochran (1977) and de Vries (1986) beside the modern literature [6-9] }
  
  \answerfin{The respective citations have been added.}  
  % --------------------------- %
  
  % 08) ++++++++++++++++++++++++++ %
  \item \textit{L43-48: I would suggest to add some words (here or in the discussion) addressing the problems arising with multi-purpose forest inventories (e.g. Corona and Marchetti 2007): The stratification is optimized for a single target variable (in most cases the volume of the growing stock) and the introduction of additional target variables becomes challenging (e.g. Ritter and Saborowski, 2014).}
  
  \answerfin{We think that the reviewer might mix up \textit{stratification} (where the sample sizes are optimized for strata \textit{prior} to the actual conduction of the inventory based on auxiliary data) with \textit{post-stratification} (where the categorical auxiliary data are used in the regression model \textit{after} the inventory has been conducted). The latter, which is the case applied in our article, does not impose any of the mentioned problems of multi-objective optimization. We thus think that including this topic does not fit well into our article since it is not affecting the methods that we are demonstrating.} 
  
%  \note{I don't think this issue concerns the work presented in our study, since optimization was not part of the study. One could mention the anticipated variance treated in the book of Mandallaz (2008). Multi-objective optimization would implies a weighting of the objectives, which is a highly political issue and probably unsolvable in my opinion. Separate investigation are probably simpler and better. Worth emphasizing that the method presented is valid for any response variable which can be used to define a local density. I also think that the reviewer exchanges stratification (where the sample sizes are optimized prior to the actual conduction of the inventory based on auxiliary data) and post-stratification (where the categorical auxiliary data are used in the regression model after the inventory has been conducted). The latter, which is the case applied in our article, does not impose any of the mentioned problems of multi-objective optimization. We thus think that including this topic does not fit well into our article since it is not affecting the methods we suggest.}
  % --------------------------- %
  
  % 09) ++++++++++++++++++++++++++ %
  \item \textit{L46-47: Double-sample procedures have not only been used in the FDI of Lower Saxony, but also in the NFIs of Canada (Gillis et al., 2010) and Switzerland (Lanz et al. 2010), and in different other inventories (e.g. Chojnacky, 1998 or  Gasparini et al., 2010).}
  
  \answerfin{We rephrased the sentence and added the suggested references.}  
  % --------------------------- %
  
  % 10) ++++++++++++++++++++++++++ %
  \item \textit{L99 “The German NFI (Bundeswaldinventur, BWI)“  I think, the abbreviation introduced in the next line becomes more meaningful when giving the German name here.}
  
  \answerfin{Has been changed.}  
  % --------------------------- %
  
  % 11) ++++++++++++++++++++++++++ %
  \item \textit{L100 “The third and most recent” }
  
  \answerfin{Has been changed.}  
  % --------------------------- %
  
  % 12) ++++++++++++++++++++++++++ %
  \item \textit{L101 '4x4km' I would prefer '4km ' 4km'}
  
  \answerfin{We would like to stick to the given notation in accordance with our previously published article (Hill et. al 2018: Combining canopy height and tree species map information for large-scale timber volume estimations under strong heterogeneity of auxiliary data and variable sample plot sizes, \textit{European Journal of Forest Research}).}  
  % --------------------------- %
  
  % 13) ++++++++++++++++++++++++++ %
  \item \textit{L102 '2x2km' same as above (2km ' 2km)}
  
  \answerfin{Same as above.}  
  % --------------------------- %
  
  % 14) ++++++++++++++++++++++++++ %
  \item \textit{L107: A short explanation of the angle count sampling concept would be helpful for readers who aren't foresters, maybe here or in a corresponding appendix.}
  
  \answerfin{We added an explanation of angle count sampling in Section 2.}  
  % --------------------------- %
  
  % 15) ++++++++++++++++++++++++++ %
  \item \textit{L110/111 “DBH” and “DBH” Please decide whether or not to use italics. Please also add the information that this refers to a height of 1.3 m, Remote Sensing has a lot of readers who are not foresters.}
  
  \answerfin{We added the information about the height. Concerning italics, we decided to use italics when first introducing a notation/term and use normal formatting when using the term afterwards.}  
  % --------------------------- %
  
  % 17) ++++++++++++++++++++++++++ %
  \item \textit{L196 Please also provide reference to the R software itself, and not only to the package. }
  
  \answerfin{The reference has been added.}  
  % --------------------------- %
  
  % 18) ++++++++++++++++++++++++++ %
  \item \textit{L273: I think this is the only occasion you use italics for "RLP".}
  
  \answerfin{We use italics when introducing a term for the first time and use normal formatting when using the term afterwards.}  
  % --------------------------- %
  
  % 19) ++++++++++++++++++++++++++ %
  \item \textit{L283: “communal forests” is most often used as a synonym for “community forests” in context of rural development policy. I think here it is rather an odd translation of the German word “Kommunalwald”. Maybe, “municipal forests” is the better translation.}
  
  \answerfin{We changed 'communal' to 'municipal' throughout the entire article.}  
  % --------------------------- %
  
  % 20) ++++++++++++++++++++++++++ %
  \item \textit{L295 "RLP" was already introduced as an abbreviation}
  
  \answerfin{'RLP' has been deleted.}  
  % --------------------------- %
  
  % 21) ++++++++++++++++++++++++++ %
  \item \textit{L287: Instead of “45 Forstämter (FA)” I would use “45 forest districts (Forstämter, FA)” }
  
  \answerfin{Has been changed, and likewise for the Forstreviere which we now call sub-districts.}  
  % --------------------------- %
  
  % 22) ++++++++++++++++++++++++++ %
  \item \textit{L289-290: If the SFI is conducted in a five year cycle, shouldn’t the management strategy be also set up for five years, instead of ten years?}
  
  \answerfin{The cycle of the stand wise inventories actually varies between 5 and 10 years and can in some cases even exceed 10 years. The stand wise inventories and the date of their conduction are planned by the state forest service. The management strategy for the stands is, however, always made for the upcoming 10 years ('mittelfristige Planung').}  
  % --------------------------- %
  
  % 23) ++++++++++++++++++++++++++ %
  \item \textit{L295 "2x2km" (see above)}
  
  \answerfin{We would like to keep the '2x2km' notation.}  
  % --------------------------- %
  
  % 24) ++++++++++++++++++++++++++ %
  \item \textit{Figure 1: "communal forests" (see above)}
  
  \answerfin{Has been corrected.}  
  % --------------------------- %
  
  % 25) ++++++++++++++++++++++++++ %
  \item \textit{Table 1: Please remove the digits (or at least reduce the number of digits). Giving values with an “accuracy” of 1/100th of a mm (i.e. 10 mm) for DBH, as an example, really does not make much sense. }
  
  \answerfin{We limited the number of digits for timber volume, DBH and height to one digit and removed the digits for stem number.}  
  % --------------------------- %
  
  % 26) ++++++++++++++++++++++++++ %
  \item \textit{Table 3: I think one digit for the percentage values should be enough.}
  
  \answerfin{Has been changed accordingly.}  
  % --------------------------- %
  
  % 27) ++++++++++++++++++++++++++ %
  \item \textit{Table 6: Please indicate for what variable point estimates are. I suppose it is volume of growing stock per area unit [m$^3$ha-1]? And again, check the number of digits.}
  
  \answerfin{Yes! We added the missing information accordingly.}  
  % --------------------------- %
  
 % 28) ++++++++++++++++++++++++++ %
  \item \textit{L454-458 I understand your intention and the necessity of removing FR-units. However, when comparing the performance of the estimators you should use the same population for all estimators. Removing the "difficult" FR-units (i.e. those with a low sampling density) only for PSMALL and EXTPSYNTH is somewhat unfair against the other estimators, and may bias the results. So, my advice is to recalculate SRS and PSYNTH for the same 321 FRs that were used for PSMALL and EXTPSYNTH. Edit: I noticed you confined your analyses to those 321 FRs later on (Fig 5 and Tab. 7), another reason to also do it here.}
  
  \answerfin{We changed the evaluation according to your suggestion and added a sentence in the paragraph above the table. The descriptive summary statistics for the one-phase SRS and the PSYNTH estimator did only differ marginally, with exception of the maximal errors which logically turned out to be lower. We also adjusted the numbers given in Section 6.2 and the graphic of Figure 3 respectively.}
  
  % --------------------------- %
  
 % 29) ++++++++++++++++++++++++++ %
  \item \textit{L517: Please use the same scaling for both x- and y-axis on the left plot. It is irritating that the 1:1 line does not cross the grid-corners.}
  
  \answerfin{Has been changed accordingly.}  
  % --------------------------- %
  
   % 30) ++++++++++++++++++++++++++ %
  \item \textit{L655-656: "Northwest German Forest Research Institution" It should be "Institute" instead of "Institution"}
  
  \answerfin{Has been corrected.}  
  % --------------------------- %
  

\end{enumerate} 
  
  
%------------------------------------------------------------------------------------------------%
\end{document}







