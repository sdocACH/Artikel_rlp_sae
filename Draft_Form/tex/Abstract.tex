
\section*{Abstract} 
\label{sec:abstract}
The German National Forest Inventory consists of a systematic grid of permanent sample plots and provides a reliable evidence-based assessment of the state and the development of Germany's forests on national and federal state level in a 10 year interval. However, the data have scarcely been used for estimation on smaller management levels such as forest districts due to insufficient sample sizes within the area of interests and the implied large estimation errors. In this study, we present a double-sampling extension to the existing German National Forest Inventory (NFI) that allows for the application of recently developed design-based small area estimation procedures that try to overcome the restriction in terrestrial sample size by enlarging the sample by a large set of model predictions. We illustrate the implementation of the estimation procedure and evaluate its potential by the example of timber volume estimation on two small scale management levels (42 and 392 forest district units respectively) in the federal German state of Rhineland-Palatinate. A LiDAR derived canopy height model and a tree species classification map based on satellite data were used as auxiliary information in an ordinary least square regression model to produce the timber volume predictions on the plot level. The results indicate that the suggested double-sampling procedure can significantly increase estimation precision on both management levels compared to estimates based on the simple random sampling (SRS) estimator: on Forstamt-level, the proportion of estimation errors below 6\% could be increased from 10\% to 60\% (90\% were below 7\% estimation error). The increase of estimation precision was even more pronounced on the Revier-level, although the errors were generally higher than on the Forstamt-level: here, the proportion of estimation errors below 10\% could be increased from 15\% to 40\%. The average decrease in variance was ... and ..., and the relative efficiency was ... and ... on Forstamt and Revier-level respectively.



% gain
% relative efficiency
% limiting factors (--> to be solved for the future?)

% Keywords:
\vspace{0.2cm} \noindent \textbf{Keywords.} National forest inventory, small area estimation, double sampling for regression within strata, cluster sampling, LiDAR canopy height model, tree species classification \vspace{-1cm}

% NOTE
% - Problem statement: Large scale inventories, such as the German NFI, provide high accuracies on national as well as federal state level. However, the uncertainty of estimations on smaller scale management levels (destricts) often turns out be unaccepably large.
% - Multi-phase sampling procedures try to overcome the lack of terrestrial information by using an additional large sample of predictions based on remote sensing data. These techniques can particularly be efficient for domains where the terrestrial sample size is considerably small (small area estimation problem).
% - The aim of this article was to 
%
%
%
















