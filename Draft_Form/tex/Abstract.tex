
\section*{Abstract} 
\label{sec:abstract}
The German National Forest Inventory consists of a systematic grid of permanent sample plots and provides a reliable evidence-based assessment of the state and the development of Germany's forests on national and federal state level in a 10 year interval. However, the data have yet been scarcely used for estimation on smaller management levels such as forest districts due to insufficient sample sizes within the area of interests and the implied large estimation errors. In this study, we present a double-sampling extension to the existing German National Forest Inventory (NFI) that allows for the application of recently developed design-based small area regression estimators. We illustrate the implementation of the estimation procedure and evaluate its potential by the example of timber volume estimation on two small scale management levels (45 and 405 forest district units respectively) in the federal German state of Rhineland-Palatinate. An airborne laserscanning (ALS) derived canopy height model and a tree species classification map based on satellite data were used as auxiliary data in an ordinary least square regression model to produce the timber volume predictions on the plot level.
The results support that the suggested double-sampling procedure can substantially increase estimation precision on both management levels: the two-phase estimators were able to reduce the variance of the SRS estimator by 43\% and 25\% on average for the two management levels respectively.

%The German National Forest Inventory (NFI) consists of a systematic grid of permanent sample plots and provides a reliable evidence-based assessment of the state and the development of Germany's forests in a 10 year interval. Although the inventory data have proved to be satisfactory on national and federal state levels, estimation on smaller management levels, such as forest districts, has been prone to large estimation errors due to insufficient sample sizes. In this study, we present a double-sampling extension to the existing German NFI that illustrates the implementation of recently developed design-based small area regression estimators and evaluate their potential via timber volume estimation on two small scale management levels (45 and 405 forest district units respectively) in the federal German state of Rhineland-Palatinate. An airborne laser scanning (ALS) derived canopy height model and a tree species classification map based on satellite data were used as auxiliary data in an ordinary least squares regression model to produce the timber volume predictions on the plot level. The results support that the suggested double-sampling procedure can substantially increase estimation precision on both management levels, reducing the variance compared to the standard one-phase estimator on average by 43\% and 25\% respectively.

% Keywords:
\vspace{0.2cm} \noindent \textbf{Keywords.} National forest inventory, small area estimation, double sampling for regression within strata, cluster sampling, LiDAR canopy height model, tree species classification \vspace{-1cm}

