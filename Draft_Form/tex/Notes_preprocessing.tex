
%% Preprocessing:

% -------------------------------------------- %
% BWI Daten:
% -------------------------------------------- %

------------------------------------------------------------------------
------ 1) Auswahl der wzp4-Plots: ------
 
 --> an 8092 Plots (2810 Cluster) in RLP wurde die wzp4 durchgeführt.
 Es wurden Plots ausgewählt welche:
# 1) - zur BWI3 gehoeren --> inve3 = 1 & 
# 2) -  auf bestocktem Holzboden liegen --> wa = 5 & 
# 3) - im begehbaren Wald liegen --> begehbar = 1 &
# ---> 8092 Plots

------------------------------------------------------------------------
------ 2) Auswahl der wzp4-Bäume pro Plot: ------
 
Für diese 8092 Plots wurden jene Probebäume ausgewählt, welche
# 4) - in der wzp4 aufgenommen wurden (wzp4-Baeume) --> av = 0
# 5) - ! alle "bestandesinternen Bäume"* (bz = 1) ! 

--> 56149 wzp4-Bäume in RLP aufgenommen (bestandesinterne Bäume), verteilt auf 8092 Plots
    (Info: weicht von  # Bäumen in Modellierungspaper ab --> hier habe ich wohl best.übergreifend ausgewertet --> 56561)

------------------------------------------------------------------------
------ 3) Berechnung der Vorratsdichte pro Plot: ------
-> Nach Winkelzählprobe (Formel geben?) Waldrand (Bestandes)-Edge-Korrektur erwähnen
-> von den 8092 Plots sind 317 sog. "zero-plots" für welche das Plotvolumen 0 ist


------------------------------------------------------------------------
------ 4) Lagekoordinaten für Plots: ------

Erwähnen weil:
-> wichtig für Vergleichbarkeit von expl. Variablen und Plotvorratsdichte
-> erwähnen: nicht wichtig für einphasige Auswertung, aber wichtig bei Erweiterung

Aufn.methode und fehlende Messungen:
-> Im Rahmen der 3. BWI (? korrekt) alle Plots in RLP mit DGPS vermessen (Median über 100 Messungen).
-> für 375 der 8092 Plots (ca. 5\%) waren die GPS Messungen leider nicht berfügbar und werden in den
   Folgeinventuren neu vermessen. Für unsere Auswertungen wurden für diese Plots die "alten" Soll-Koordinaten" übernommen

Abweichungen zwischen alten Soll-Positionen und neuen GPS-Positionen:
-> ein Vergleich der alten Soll-Koordinaten und der DGPS IST-Koordinaten ergibt, dass
   eine Abweichung von maximal 25m für 90\% der Plots (das sagt aber noch nichts über
   die Genauigkeit der DGPS-Koordinaten aus).
   
Genauigkeit der DGPS-Messung:  
-> schwierig abzuschätzen
->  90\% der DGPS vermessenen Plots haben einen "Idealen" oder "Exzellenten" HDOP-Wert (HDOP aufgezeichnet
    bei Messung, aber keine Metadaten zur Lagegenauigkeit in [m])
-> Lambrecht et. al schätzen aufgrund ihrer Koregistrierung im Nationalpark HHW, dass 80\% der DGPS-Messungen 
   eine Abweichung zwischen 1.4 und 8.7 m haben
-> eigene Analyse für ganz RLP ergibt, dass die Abweichungen zwischen Soll- und IST-Koordinaten unabhängig von
   vom HDOP und der Anzahl Messungen ist

Plausibilitätsprüfung(en):
1) Fehlerhafte Datenablage: für 2 Plots ändert sich fälschlicherweise die Trakt-ID
2) Trakstruktur: für 4 Plots ist die Traktstruktur nicht stimmig --> siehe Slides
 --> 6 unplausible Plots wenn man DGPS-Messung vertraut --> diese wurden durch die alten Soll-Koordinaten ersetzt
 
 
==> insgesamt wurden also für 375 + 6 = 381 der 8092 Plots die alten Soll-Koordinaten verwendet (4.7\%)


% ------------------------------------------------ %
% Erweiterung des terrestrischen Stichprobennetzes:
% ------------------------------------------------ %

Das terrestrische Stichprobennetz (2x2km) wurde auf 500x500 meter (4-fach) verdichtet

Regeln:
1) Alle Punkte müssen innerhalb der obersten definierten Management-Fläche (Forstämter) liegen
2) Der Wald-Nichtwald-Entscheid muss bei Anwendung des mehrphasigen Verfahrens sowohl auf die terrestrischen als auch
   die "synthetischen" Inventurpunkte (sprich: ALLE Punkte) angewendet werden können. Daher werden alle Punkte anhand der
   Geodatenlayer (Woefis, Atkis) maskiert. --> führt zu variabler Anzahl Plots pro Trakt (entspricht Theorie)

Restriktion auf Wald-Layer:

aufgetretene Effekte:
 -> 2 terrestrische Plots ("63718_3", "60767_4") liegen ausserhalb der foa-geometry --> Datensatz von 8092 auf 8090 reduziert
 -> es liegen zusätzlich 179 terrestrische Proben ausserhalb der beachteten Gesamtwalddefinition union(Atkis, woefis)-Waldlayers 
    (179 / 8092 = 2\%)
 -> für unsere Woefis"wald"-Auswertung: 5791 terrestrische Plots (2055 Cluster) fallen in den Woefis-Wald (5791/8090 = 71.5\%, entspricht ziemlich gut des von der BWI angegebenen Anteils an Gemeinde- und Staatswald:73\%) 
 
 -> Stichpobenzahl insgesamt für woefis-wald: 96854 plots (33365 Cluster)
    



% -------------------------------------------- %
% Aufbereitung der Fernerkundungsdaten
% -------------------------------------------- %

--> folgt Beschrieb im Modellierungsartikel

--> missing data erwähnen

--> Erwähnen: Im Gegensatz zur Modellierungs-Studie, in der wir die 'zero-plots' wegggelassen haben, schliessen wir diese hier ein 
     (Grund: bei Berechnung der Punktschätzungen in der angewendeten Software (Verweis) wird die 'zero-mean-property' (zmp) der Residuen 
             in F und G vorrausgesetzt (dadurch entfällt der Residuenterm). Die zmp wäre ist hier dann aber nur gewährleistet, wenn wir das Modell
             mit ALLEN terrestrischen Daten fitten, welche später auch in die Schätzungen eingehen)
      das Weglassen der zero-plots in der Schätzung kommt aber natürlich nicht in Frage! Also ist das Vorgehen hier die einzige Möglichkeit
      
    ==> es ergibt sich durch Einbeziehung der zero-plots ein ganz leicht anderes adjusted R2 (0.49 anstatt 0.48), lustigerweise sogar etwas höher als vorher.
        Erwähnen, dass die Aussagen aus dem Modellierungspaper davon nicht betroffen sind.     



% -------------------------------------------- %
% Reg.Modell und Ableitung Hilfsvariablen
% -------------------------------------------- %

Berechnungszeiten für Woefis-Wald: (96854 plots)
-> Berechnung Supports: Berechnungszeit für woefis-wald: 67 Minuten
-> Berechnung CHM-Metriken: 1 Minute
-> Berechnung TREESPECIES-Metriken: 6 Minute
-> Berechnung FRACCOVER-Metriken: 4 Minuten
-> Berechnung wgbz-Metriken: 610.96 Minuten





