\section{Discussion} %article
\label{sec:Dis}

% ------------------------ %
\subsection{Performance of estimators}

The aim of this study was to investigate the performance of model-assisted design-based estimators for small area estimation of mean standing timber volume on two spatial forest management levels in Germany. It was of particular interest to gather information about the estimation error levels that can be attained using German NFI data that is characterized by low sampling intensities in the area of interests. To address these research questions, we applied the SRS, the \psmall{} and the \extpsynth{} estimators for cluster sampling to two forest management levels consisting of 45 and 405 small area units respectively in the German state of Rhineland-Palatinate.\par

Our study showed that on both small area levels, the \psmall{} and the \extpsynth{} estimators generally led to a substantial reduction in estimation error compared to the standard one-phase SRS estimator. On the upper management level (FA districts), \psmall{} and \extpsynth{} produced estimation errors smaller than 5\% for 73\% of the small areas compared to only 17\% under the SRS estimator. The same level of precision could not be achieved on the lower management level (FR districts) primarily due to substantially smaller terrestrial sample sizes. However, in 95\% of the FR units, the estimation errors could be limited to 20\% compared to 40\% under SRS. A pairwise comparison of the confidence intervals revealed that the estimators did not produce significantly different point estimates. The much smaller estimation errors of the \psynth{} estimator 
reflected the fact that it does not try to correct for potential bias in the point estimate which can lead to overly optimistic estimation errors and confidence intervals. One should thus prefer the unbiased estimates of \psmall{} or \extpsynth{}.\par

For several FR units, it was observed that the \psmall{} and the \extpsynth{} estimator can occasionally produce larger variances than the SRS estimator. It is important to note that this is in perfect agreement with the theory of both two-phase estimators and can theoretically appear if the residual variance in the small area, which generally constitutes the dominating part of the two-phase variance, turns out to be much higher than the variance of the terrestrial data in the small area. The empirical findings of our study suggest that such cases can particularly occur if moderate or poor model fits within a small area are combined with small terrestrial sample sizes ($\leq 5$) in the small area. A closer look on these small areas thus might reveal the reason for the poor prediction performance and help to improve the model fit. Nonetheless, it should be kept in mind that small terrestrial sample sizes can also cause the SRS estimator to not reflect the actual variation of the local density within a small area. In this case, the two-phase variance estimate might be larger but more realistic. Whereas a visual analysis of aerial images, remote sensing data or stand maps might give some further evidence for or against this hypothesis, a definite proof is practically infeasible.\par

We were also able to empirically confirm that the \extpsynth{} estimator generally produces slightly smaller variances and estimation errors than the \psmall{}. This is most probably caused by marginally smaller model residuals due to the intercept adjustment to the terrestrial data in the small area unit, which is primarily a means to ensure the zero mean residual property of the \extpsynth{}. However, our analysis indicated that the difference between the two estimators is negligible for sample sizes $\geq$ 10 due to their asymptotic equivalency. Furthermore, one or more clusters not entirely included in the small area unit did not have a notable impact on the estimates of \extpsynth{} when the terrestrial sample size was more than 6. However, there was a slight but statistically significant tendency to be over-optimistic for sample sizes between 4 and 6. More empirical evidence must be gathered before generalizing this as a rule of thumb for the application of the \extpsynth{} under cluster sampling. It thus seems recommendable to calculate both \psmall{} and \extpsynth{}, and subsequently compare their results. If no suspicious deviations occur, we consider the \extpsynth{} as the estimator of choice.\par

% ------------------------ %
\subsection{Auxiliary data}

The auxiliary data used in our study were derived from two remote sensing sources, i.e. an ALS canopy height model and a tree species classification map. Likewise in many similar studies, the ALS mean canopy height proved to be the explanatory variable with highest predictive power. However, the large time-gaps of up to 10 years between the ALS acquisition and the terrestrial survey date caused the substantial introduction of artificial noise in the data. Whereas a post-stratification to the ALS acquisition years was an effective means to counteract the implied residual inflation, several leverage points were unambiguously caused by the temporal asynchronicity. Undetectable forest loss during the gap between the ALS acquisition and the NFI was also likely a cause for high residual variance in some small areas compared to the terrestrial data variance, which subsequently led to higher variances than the SRS estimator. As opposed to the ALS data, the availability of a country-wide tree species classification map has yet been unique among all German federal states. Whereas the study of \citet{hill2017a} already showed that the tree species information was able to improve the model fit, it has yet not been used to its full potential. One reason for this was the impossibility of modeling individual tree species within each ALS acquisition year, which would add further explanatory power. Another reason was the lack of available satellite data for classification in some parts of the country, which led to missing values in the inventory data and restricted 19 FR units to a simpler regression model. Promising steps with respect to more up-to-date canopy height information have already been made, as the topographic survey institution of RLP will from this year on provide a country-wide canopy height model derived from aerial imagery acquisitions. These campaigns will in the future be conducted in a two-year period and allow to derive canopy height information matching the dates of terrestrial forest inventories. A study of \citet{kirchhoefer2017} recently indicated that similar model performance for German NFI data can be achieved using such imagery-based canopy height models. Due to the improved coverage and repetition rate of the Sentinel-2 satellite \citep{sentinel2}, the tree species classification map will in the future be updated each year. We consider these alternative auxiliary data sources to also solve the problem of missing explanatory variables at inventory plots. One could also make use of the exhaustive information within the two-phase estimators by using the true the auxiliary means \citep{mandallaz2013a, mandallaz2013b}, which could further decrease estimation errors. Previous studies of \citet{mandallaz2013b} however showed that given a reasonable large sample size of the first phase, the differences in the estimation error are usually small. With respect to the substantial improvements in the temporal synchronicity between auxiliary and terrestrial inventory data, we consider the demonstrated double-sampling approach also to be very efficient for change estimation \citep{massey2015b}.\par




%% auxiliary information:
% - realistically represent current state of ALS availability (see BW), but also tspec info as a specialty
% - still missingness in the auxililary data
% - future perspective (BDOM, tspec)
% - conclusion for estimators: 
%   o a) using exhaustive aux. variable info in estimators
%   o b) increase in model accuracies due to more flexible OLS models (*) *: this might also solve the problem of hight residual variations in some FR units (SRS <-> 2p)
%   o c) demonstrated procedures can then also be used for change estimation (without the inconvenience of time gaps)

%% conistency of sampling frame:
% - (**)

% ---------------------------------------------------------------------- %
% NOTES:
%
% begin with general results:
%
% - extension to change estimation
%   o BDOM more frequently updated, max. 1 year time gap between CHM and BWI3 survey
%   o also treespecies classific. frequently updated using sentinel 2 data
%   o the gain in model accuracy will certainly considerably increase if the ALSyear stratification becomes unnecessary Value increase also in combination with
%     further explanatory variables such as growing conditions
%
%
% (**):
% It should however be mentioned that a discrepancy between the terrestrially made forest/non-forest decision and that based on the polygon stand layer can lead to missing responses in the terrestrial sample $s_2$. This occurs if a terrestrial plot $x_l \in s_2$ does not provide the terrestrial response variable although its plot center would theoretically fall within the boundaries of the stand layer. For these cases, it is important to note that according to the double sampling estimator described in Section \ref{sec:SAestimators}, the plots of a cluster can either only belong to the sample $s_2$ or $s_1$. The missing terrestrial plot information $Y(x)$ can thus not be replaced by a prediction $\hat{Y}(x)$, i.e. by establishing a plot $x \in s_1$ instead, and has thus to be treated as a non-response.\par



% - CONCLIUSION: 
%   o transfer of the suggested procedure to FDI inventories: maybe possible to get qunatified data for stands with acceptable estimation errors? (would be of high interest)
%   o the decision wether the errors are acceptable cannot be answered here but (in our opinion) depend on the question whether ... decisions can be made based on the provided Confidence Intervals.
%   o in a next step, we will investigate whether the derived confidence intervals are sufficient to be used as a validation for the stand-wise inventories (Kuliesis et al.)
% 
%
% - Kuliesis et al.:  --> "NFI [provide reliable estimates for the most important attributes and] can be used 
%                          as a tool for validation of all other inventories", i.e. stand-wise inventories
%                     ==> in our case, we can say that we illustrated how the German NFI data can now be used
%                         for this purpose, i.e. providing of reliable data to validate the stand-wise inventories
%


