\appendix
\section{Appendix}
\label{sec:appendix_saerlp}

% -------------------------------------- %
\subsection*{R-squared on cluster level}

The R$^2$ on the cluster level is calculated using the number of plots $M(x)$ of each cluster in order to weight for the varying number of plots on which $Y_c(x)$ and $\hat{Y}_{c}(x)$ are based on. 

\begin{equation*}\label{eq:R_squared_clust}
R^2 =  \frac{\sum_{x \in s_2} \Big(\frac{M(x)}{\bar{M_2}}\Big)^2 \Big(\hat{Y}_{c}(x)-\hat{\bar{Y}}_c\Big)^2}{\sum_{x \in s_2} \Big(\frac{M(x)}{\bar{M_2}}\Big)^2 \Big(Y_{c}(x)-\hat{\bar{Y}}_c\Big)^2}
\end{equation*}

\noindent $Y_c(x)$ and $\hat{Y}_{c}(x)$ are the predicted and observed local densities on the cluster level calculated according to equations \ref{eq:locdens_clust} and \ref{eq:globresids}. $\hat{\bar{Y}}_c$ is the estimated sample mean corresponding to the weighted mean over all observed local densities on the cluster level (equ. \ref{eq:srs_est_clusters}).\par

% -------------------------------------- %
\subsection*{RMSE on cluster level}

The same weights $M(x)$ are also applied to calculate the RMSE on the cluster level. $n_2$ is the number of clusters used in the modeling frame.

\begin{equation*}\label{eq:RMSE_clust}
RMSE= \sqrt{\frac{1}{n_{2}}  \sum_{x \in s_2} \Big(\frac{M(x)}{\bar{M}_2}\Big)^2 \Big(\hat{Y}_{c}(x)-Y_c(x)\Big)^2}
\end{equation*}

\noindent The \textit{relative} or \textit{normalized} RMSE is calculated by setting the absolute RMSE in relation to the estimated sample mean $\hat{\bar{Y}}_c$:

\begin{equation*}\label{eq:NRMSE_clust}
RMSE[\%] = \frac{RMSE}{\hat{\bar{Y}}_c}
\end{equation*}

\noindent Generally note that the weights $\frac{M(x)}{\bar{M_2}} \equiv 1$ if the number of plots per cluster is constant.
