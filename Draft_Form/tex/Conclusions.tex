\section{Conclusion}
\label{sec:concl}

The study led to two major conclusions: (1) the \extpsynth{} and \psmall{} estimator generally achieved substantially smaller estimation errors on the two investigated forest district levels compared to the SRS estimator. The demonstrated double-sampling procedure thus constitutes a major contribution to an increase in value of the existing German NFI data on the federal state level. However, it is not possible to conclude from our study results alone whether the realized error levels are already acceptable in order to support forest planning decisions. Thus, further investigations are necessary in close cooperation with the forest authorities. A first study will concentrate on testing the \extpsynth{} and \psmall{} confidence intervals as a validation source for the stand-wise inventories. (2) Despite the quality restrictions in the ALS data and the tree species map, the two data sources were found to be well suited to model the mean timber volume on plot and cluster level. With respect to frequently updated aerial canopy height models and tree species maps, it will thus be of hight interest to investigate the model and estimation accuracies that can be expected for future applications. In this framework, the incorporation of additional auxiliary data and the extension to change estimation seem the reasonable next steps to be explored towards an operational implementation of the demonstrated double-sampling procedure.


%The application of the design-unbiased \extpsynth{} and \psmall{} estimator should be preferred over synthetic estimations whenever possible, since they also account for the model inaccuracies and thus provide unbiased estimates as well as more realistic confidence intervals.
%
%
%
%
%We draw three major conclusions from our study: (1) our analyses strongly indicated that the acquisition of auxiliary data close to the date of the terrestrial survey is a key factor to achieve good model accuracies. Particularly for large-scale inventory applications, this requirement is often difficult to meet. In such cases, we consider that the proposed method of including quality information about the auxiliary data in a prediction model can be an effective technique for improving the prediction accuracy. Ongoing studies investigate whether this modelling technique can also lead to smaller estimation errors of design-based estimators. (2) Our study also indicated that the relationship between the field measured timber volume and remote-sensing derived height information is tree species specific. We expect that using the tree species information in a timber volume model would even lead to higher prediction accuracies when combined with explanatory variables that can further explain the variation within each tree species group, such as bioclimatic growing conditions, soil properties and stand density on the plot level. (3) We consider the demonstrated calibration technique to be a valuable method for future studies where an external tree species map (i.e. the map was not created for the specific study objective) is used in prediction models. The application of a calibration model can also be transferred to any error-prone explanatory variable and be a simple means to clean the data set from noise and thus increase the model accuracy.


% - CONCLIUSION: 
%   o transfer of the suggested procedure to FDI inventories: maybe possible to get qunatified data for stands with acceptable estimation errors? (would be of high interest)
%   o the decision wether the errors are acceptable cannot be answered here but (in our opinion) depend on the question whether ... decisions can be made based on the provided Confidence Intervals.
%   o in a next step, we will investigate whether the derived confidence intervals are sufficient to be used as a validation for the stand-wise inventories (Kuliesis et al.)

% - The existing German NFI data can provide substantially smaller estimation errors on forest district levels when used in the proposed double-sampling estimators. However, the question which errors are acceptable to support operational forest planning cannot be answered purely based on the absolute error values. In our opinion, it depends on the question whether the derived confidence intervals a) allow for more profound decisions between forest planning alternatives. 

% - prefer model-assisted estimators to synthetic estimators if one can afford --> because design-unbiased, they take the model inaccuracies into account and thus provide unbiased point estimates and more realsitic confidence intervals.
