\section{Conclusion}
\label{sec:concl}

The study led to two major conclusions: (1) the \extpsynth{} and \psmall{} estimator generally achieved substantially smaller estimation errors on the two investigated forest district levels compared to the SRS estimator. The demonstrated double-sampling procedure thus constitutes a major contribution to an increase in value of the existing German NFI data on the federal state level. However, it is not possible to conclude from our study results alone whether the realized error levels are already acceptable in order to support forest planning decisions. Thus, further investigations are necessary in close cooperation with the forest authorities. A first study will concentrate on testing the \extpsynth{} and \psmall{} confidence intervals as a validation source for the stand-wise inventories. (2) Despite the quality restrictions in the ALS data and the tree species map, the two data sources were found to be well suited to model the mean timber volume on plot and cluster level. With respect to frequently updated aerial canopy height models and tree species maps, it will thus be of hight interest to investigate the model and estimation accuracies that can be expected for future applications. In this framework, the incorporation of additional auxiliary data and the extension to change estimation seem the reasonable next steps to be explored towards an operational implementation of the demonstrated double-sampling procedure.